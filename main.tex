\documentclass[a4paper, twocolumn, 10pt]{article}
\usepackage[top=2.5cm, bottom=2.5cm, left=2cm, right=2cm]{geometry}

\setlength{\columnsep}{1cm}
\usepackage{titling}
\setlength{\droptitle}{-1.25cm}
\usepackage{float}
\usepackage{graphicx} % Required for inserting images
\usepackage{times} % Use Times New Roman font
\usepackage{authblk} % Required for authors and details
\usepackage{xcolor} % Required for specifying colors
\usepackage{sectsty}
\usepackage{diagbox}
\usepackage{authblk}
\usepackage{wrapfig}
\usepackage{graphicx}
\allsectionsfont{\raggedright} % align sections title to the left$
\sectionfont{\fontsize{10}{12}\selectfont}

\usepackage{titlesec}
\usepackage[round]{natbib}
\usepackage{microtype}

% always as a last one package
\usepackage[pdfsubject={Lastname F., Lastname2 F2. and Lastname3 F3.: Example Paper: a General Model of Information Transfer},
pdfauthor={Lastname F., Lastname2 F2. and Lastname3 F3.},
pdftitle={Lastname F., Lastname2 F2. and Lastname3 F3.: Example Paper: a General Model of Information Transfer},
pdfkeywords={Model; pi in the C-Ky; X-Y-Z Analysis}]{hyperref}

\titleformat{\subsubsection}{\normalfont\itshape}{\thesubsubsection}{1em}{}
\titleformat{\subsection}{\normalfont\bfseries}{\thesubsection}{1em}{}

\makeatletter
\renewcommand\@biblabel[1]{}

\renewenvironment{thebibliography}[1]
     {\section*{\refname}%
      \@mkboth{\MakeUppercase\refname}{\MakeUppercase\refname}%
      \list{\@biblabel{\@arabic\c@enumiv}}%
           {\settowidth\labelwidth{\@biblabel{#1}}%
            \leftmargin\labelwidth
            \advance\leftmargin20pt% change 20 pt according to your needs
            \advance\leftmargin\labelsep
            \setlength\itemindent{-20pt}% change using the inverse of the length used before
            \@openbib@code
            \usecounter{enumiv}%
            \let\p@enumiv\@empty
            \renewcommand\theenumiv{\@arabic\c@enumiv}}%
      \sloppy
      \clubpenalty4000
      \@clubpenalty \clubpenalty
      \widowpenalty4000%
      \sfcode`\.\@m}
     {\def\@noitemerr
       {\@latex@warning{Empty `thebibliography' environment}}%
      \endlist}
\renewcommand\newblock{\hskip .11em\@plus.33em\@minus.07em}
\makeatother

\pagenumbering{gobble} % Removes page numbers from all pages


\title{\textls[6]{
\fontsize{14}{0}\selectfont\lsstyle\textbf{
EXAMPLE PAPER: A GENERAL MODEL OF INFORMATION TRANSFER
}}
\vspace{-0.5em}}


\author[1]{Firstname Lastname}
\author[1]{Firstname2 Lastname2}
\author[2]{Firstname3 Lastname3}
\affil[1]{Department of Computer Science, University X, e-mail: ecms@scs-europe.net}
\affil[2]{Department of Computer Science, University Y, e-mail: ecms@scs-europe.net}

\date{}
\begin{document}

\maketitle

\section*{KEYWORDS}
Model, $\pi$ in the C-Ky, X-Y-Z Analysis.

\section*{ABSTRACT}

Begin each paper with an abstract (100-200 words) that summarises the topic and important results presented in the paper. It should start in the left column, approximately 7cm from the top edge of the paper, and 2cm from the left edge. Include the abstract heading as shown. Use the bold version of your font and type in caps. Skip a line space, then begin the abstract at the left margin. The abstract should preferably not contain formulas, pictures, or references.

\section*{PREPARING THE REMAINING PAGES}
For the remaining pages follow the general guidelines below:

\section*{MAJOR HEADINGS}
Type in capitals, beginning flush with left-hand margin. Use a bold font. Skip half a line space, then begin.

\subsection*{Subheadings}
Capitalise the first letter of each word, beginning flush with left-hand margin. Use a bold font. Skip half a line space, then begin.

\subsubsection*{Secondary Subheadings}
Try to avoid secondary subheadings as much as possible. These should not be necessary in a 7- or 10-page paper. Use the standard font or the italic version of the font (in this case Times Italic 10). Capitalise the first letter of each word. Text follows on the next line, do not skip a line space.

\section*{MATHEMATICAL NOTATIONS AND EQUATIONS}
Each displayed equation should be proceeded and followed by a half blank line. Display only the most important equations and number only the displaced  equations that are explicitly referenced within the text. Within the display enclose the equation number in parentheses and  place it flush with the right-hand margin of the column, for example:

\begin{equation}
x(t) = A(t)x(t) + w(t) + u(t) \label{eq:1}
\end{equation}

Within the text each reference to an equation number should also be enclosed in parentheses as in this reference to Equation (\ref{eq:1}) above. The equation should preferably also be 10 points.

\section*{TABLES AND ILLUSTRATIONS}
All artwork, captions, graphics, and tables will be reproduced exactly as you submit them. To ensure quality of reproduction, do the following test. \textbf{Make a photocopy of a photocopy of the original. If this 2nd copy is still legible, it will be faithfully reproduced in the Proceedings.}
\\
\\
Figures and tables should be centred within a column. A figure or table that is wider than one column may be centred between the columns, but it should not extend beyond the column edges A figure or table that is wider than both columns should be set landscape. The top of the figure or table should be aligned with the left-column edge, and it should occupy the entire page (but not outside the  24,7cm by 17cm rectangle).
Figures and tables are numbered sequentially, but separately using Arabic numerals  Figures should appear following the paragraph on which the figure is first referenced  Centre the figure number and the caption under the figure using the following format.

\begin{figure}[H]
  \centering
  \includegraphics[width=\linewidth]{images/figure1.png}
  \caption{Capitalise Caption with No-Period.}
  \label{fig:sample}
\end{figure}

 
Tables should appear in the document following the paragraph in which the table is first referenced. Centre the table number and the caption above the table using the following format.

\begin{table}
  \centering
  \caption{Capitalise Caption with No-Period}
  \label{tab:table-label}
\begin{tabular}{|c|c|c|c|}
\hline
\backslashbox{mg}{R} & 3    & 5     & 10   \\ \hline
5                  & 0,10 & 0,20  & 0,05 \\ \hline
10                 & 0,15 & 0,40  & 0.10 \\ \hline
20                 & 0,20 & 0,80  & 0,15 \\ \hline
30                 & 0,25 & 1,60  & 0,20 \\ \hline
40                 & 0,30 & 3,20  & 0,25 \\ \hline
50                 & 0,35 & 6,40  & 0,30 \\ \hline
60                 & 0,40 & 12,80 & 0,35 \\ \hline
\end{tabular}
\end{table}

Captions for figures and tables should be written in heading style. That means only important words are capitalised.
\\
\\
\textbf{If you include photographs or screen-dumps insert them electronically or paste electronically them into place. You can use colour illustrations in your text. When using Microsoft Word™ do \underline{not anchor} your pictures,  graphs or illustrations, to a particular location in your text, as when sent electronically this position can shift. Colour will be reproduced only on the CD-ROM version of your paper – make sure that the black/white version of the paper is also readable.}

\section*{FOOTNOTES AND ENDNOTES}
Do not use footnotes or endnotes, instead incorporate such material into the text directly or parenthetically. If you want to acknowledge a grant or support, include an extra acknowledgements section at the end of the paper.

\section*{REFERENCES}
In text, references should be cited by the last name of the author and the year of publication, all in parentheses. The Reference List should be organised alphabetically by the name of the author, followed by the author's initials, year of publication, and other complete information about the published work. It may not be numbered. Only references that may be readily obtained should be cited in the list. Others may be referred to as "personal communication" in the text. In the reference list, multiple entries with the same author are arranged chronologically. Italicise the name of the publication in which the article is found, or the title itself if a separate publication. For laboratory, company, or government reports, all information on how to obtain the report should be included. For Ph.D. and M.S. theses, the institutions granting  the degree should be given. References to proceedings must include the full name of the proceedings, how to obtain it, year of publication, and page numbers of article cited. A reference to part of a book must include the range of pages in which the material is cited. Names of periodicals must be written out in full, and the range of pages cited.  Citing websites as references should not be done.

\subsection*{Examples of References}
\textit{In text:} \\
\citep{balci1985} -- one author\\
\citep{balci1981} -- two authors \\
\citep{felker1980} -- more than two authors \\
\citep{Hawking1982_1} -- a trailing lowercase letter should distinguish multiple papers by the same author(s) published during a single year (also in the references section)\\
\citep{balci1981, Hawking1982_2, gass1978} -- more references
\\
\\
\textit{In reference List (a 9-point font is standard for the reference list):} \\
The examples below show what a reference should look like in a journal (\citealt{balci1981}), for a book (\citealt{felker1980}), for a book chapter (\citealt{balci1983}), in a conference proceeding (\citealt{gass1978}), for a special publication (\citealt{nbstdocs1976}), for a technical report (\citealt{balci1985}), and for an internal report (\citealt{inglehart1983}).


\bibliographystyle{plainnat}
\setlength{\itemindent}{0pt} % Set the first line indent to 0

\bibliography{bibliography.bib}


\section*{NUMBERING PAGES AND PROCESSING}
\textbf{DO NOT USE PAGE NUMBERS}. Final page numbers will be inserted by the publisher. If page numbering is on, turn it off.

\section*{HEADERS AND FOOTERS}
Do \underline{\textbf{not}} use or set \underline{\textbf{any}} headers or footers in the paper.

\section*{GENERAL GUIDELINES TO IMPROVE THE CONTENTS OF YOUR FINAL PAPER}

\textbf{Please take note of these comments before writing your paper}

\begin{enumerate}
    \item Always state the relation with the simulation or related field throughout your \textbf{entire} paper. The relation can be threefold: your research is on simulation itself, your research \textbf{using} simulation in another field, or your research contributes to the field of simulation.

    \item If your paper describes \textbf{a method} or \textbf{a technique}, always give examples of the \textbf{use} of that method or technique. If possible, include (references to) empirical evidence that your method or technique works the way it is supposed to.

    \item If your paper describes\textbf{ an application example}, always point out the \textbf{methods} and \textbf{techniques} you used to get the results that are described in the paper. Add screen-dumps if possible. Literature references might be sufficient.

    \item Give an indication of the current \textbf{stage} your work or research is in. Is it still preliminary? Is there a commercially available product on the market? Is your work being applied by others or in other domains?

    \item Start your paper with a brief \textbf{abstract}. Do not include material in the abstract that will not be described in the paper.

    \item Always indicate the \textbf{novel aspects} that will be covered in the paper with respect to earlier work of yourself and of others. In what way does your contribution differ from what has been done before?

    \item Make sure your \textbf{introduction} is appropriate. An introduction should give the general background for your research, and references to general literature on the subject for interested readers.
    
    \item Make sure your \textbf{literature} references are correct, and up-to-date. Others will use your paper to find more information on the subject your paper is about. \textbf{Do not only use your own references}, but also refer to work done by others.

    \item End your paper with \textbf{conclusions} and \textbf{further research} and recommendations. Your conclusions must be based on the material that can be found in the paper!

    \item If you are not a native English speaker, have your English \textbf{corrected} before you send in your final paper. If the use of the English language is below standards, your paper might be rejected. The text should be written in \textbf{British English}.

\end{enumerate}

\section*{AUTHOR BIOGRAPHIES}

Please include a brief biography (no more than 300 words) of all authors at the end of the manuscript. The section heading is \textbf{AUTHOR BIOGRAPHIES}. This allows the viewing and reading audience to become familiar with the background of the authors, thus giving the paper greater impact and validity. 
Start the paragraph devoted to each author's name, without indentation. In boldface \textbf{FULL CAPITALS}. If there are multiple authors, separate each paragraph with a blank line. Authors are encouraged to include email and web address in the last line of the biography. 
If space permits, a small picture of 1.5 x 2 cm can be included as shown below. \\


\begin{wrapfigure}{l}{0.8cm} 
    \includegraphics[width=1.5cm,height=2cm,clip,keepaspectratio]{images/john_smith.png}
\end{wrapfigure}
\textbf{JOHN J. SMITH} was born in Antwerp, Belgium and went to the Free University of Brussels, where he studied chemical technology and obtained his degree in 1973. He worked for a couple of years for the Brussels Chemical Company Actin before moving in 1986 to the University of Anytown where he is now leading a large research group in the field of simulation for X-Y-Z Analysis. His e-mail address is: \texttt{JJSmith@anytown.com} and his web page can be found at \texttt{http://www.anytown.com/$\sim$smith}. \\ \\


\textbf{If you have any questions regarding the preparation of manuscripts, it is best to have them answered now - contact the ECMS Office for any clarification.}

\end{document}